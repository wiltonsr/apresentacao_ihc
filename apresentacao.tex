\documentclass[10pt]{beamer}
\usetheme[
%%% options passed to the outer theme
%    hidetitle,           % hide the (short) title in the sidebar
%    hideauthor,          % hide the (short) author in the sidebar
%    hideinstitute,       % hide the (short) institute in the bottom of the sidebar
%    shownavsym,          % show the navigation symbols
%    width=2cm,           % width of the sidebar (default is 2 cm)
%    hideothersubsections,% hide all subsections but the subsections in the current section
%    hideallsubsections,  % hide all subsections
%    left                % right of left position of sidebar (default is right)
  ]{Aalborg}

% If you want to change the colors of the various elements in the theme, edit and uncomment the following lines
% Change the bar and sidebar colors:
%\setbeamercolor{Aalborg}{fg=red!20,bg=red}
%\setbeamercolor{sidebar}{bg=red!20}
% Change the color of the structural elements:
%\setbeamercolor{structure}{fg=red}
% Change the frame title text color:
%\setbeamercolor{frametitle}{fg=blue}
% Change the normal text color background:
%\setbeamercolor{normal text}{bg=gray!10}
% ... and you can of course change a lot more - see the beamer user manual.

\usepackage[utf8]{inputenc}
\usepackage[brazil]{babel}
\usepackage[T1]{fontenc}
% Or whatever. Note that the encoding and the font should match. If T1
% does not look nice, try deleting the line with the fontenc.
\usepackage{helvet}

% ---
% Extras
% ---
\usepackage{booktabs}

% colored hyperlinks
\newcommand{\chref}[2]{%
  \href{#1}{{\usebeamercolor[bg]{Aalborg}#2}}%
}

\title[The Aalborg Beamer Theme]% optional, use only with long paper titles
{The Aalborg Beamer Theme}

\subtitle{v.\ 1.3.0}  % could also be a conference name

\date{\today}

\author[Jesper Kjær Nielsen] % optional, use only with lots of authors
{
  Jesper Kjær Nielsen\\
  \href{mailto:jkn@es.aau.dk}{{\tt jkn@es.aau.dk}}
}
% - Give the names in the same order as they appear in the paper.
% - Use the \inst{?} command only if the authors have different
%   affiliation. See the beamer manual for an example

\institute[
%  {\includegraphics[scale=0.2]{aau_segl}}\\ %insert a company, department or university logo
  Dept.\ of Electronic Systems\\
  Aalborg University\\
  Denmark
] % optional - is placed in the bottom of the sidebar on every slide
{% is placed on the bottom of the title page
  Department of Electronic Systems\\
  Aalborg University\\
  Denmark

  %there must be an empty line above this line - otherwise some unwanted space is added between the university and the country (I do not know why;( )
}

% specify the logo in the top right/left of the slide
\pgfdeclareimage[height=1cm]{mainlogo}{figuras/aau_logo_new} % placed in the upper left/right corner
\logo{\pgfuseimage{mainlogo}}

% specify a logo on the titlepage (you can specify additional logos an include them in
% institute command below
\pgfdeclareimage[height=1.5cm]{titlepagelogo}{figuras/aau_logo_new} % placed on the title page
%\pgfdeclareimage[height=1.5cm]{titlepagelogo2}{figuras/aau_logo_new} % placed on the title page
\titlegraphic{% is placed on the bottom of the title page
  \pgfuseimage{titlepagelogo}
%  \hspace{1cm}\pgfuseimage{titlepagelogo2}
}


% ---
% Início do documento
% ---
\begin{document}
% the titlepage
{\aauwavesbg
\begin{frame}[plain,noframenumbering] % the plain option removes the sidebar and header from the title page
  \titlepage
\end{frame}}
%%%%%%%%%%%%%%%%
\input{fixos/agenda}

% ------------------------------------------------------------
% Inserir slides aqui
% ------------------------------------------------------------
\section{Introduction}
% motivation for creating this theme
\begin{frame}{Introduction}{}
\begin{block}{Why the Aalborg beamer theme?}
  \begin{itemize}
    \item<1-> In August 2010, I gave a presentation at the European Signal Processing Conference (EUSIPCO) here in Aalborg. For that purpose, I created the AAU sidebar beamer theme.
    \item<2-> Since there was no Aalborg University (AAU) branded beamer theme, I published the theme after the conference on my website so that other researches and students at AAU could use the theme for their presentations.
    \item<3-> I have received a lot of positive feedback. To my surprise, several people not affiliated with AAU have used the theme despite the heavy AAU branding.
    \item<4-> To make the theme more usable to people not affiliated with AAU, I have decided to create this theme called \alert{Aalborg} which is very similar to the AAU sidebar theme, but it does not come with the heavy AAU branding.
  \end{itemize}
\end{block}
\end{frame}
%%%%%%%%%%%%%%%%

\subsection{License}
% the license
\begin{frame}{Introduction}{License}
  \begin{itemize}
    \item<1-> The AAU logo is covered by copyright rules. I have used the logo from \chref{http://aau.designguides.dk}{http://aau.designguides.dk}. As long as you use the theme for making presentations in connection with your work at AAU, you are allowed to use the AAU logo.
    \item<2-> The rest of the theme is provided under the GNU General Public License v. 3 (GPLv3). This basically means that you can redistribute it and/or modify it under the same license. For more information on the GPL license see \chref{http://www.gnu.org/licenses/}{http://www.gnu.org/licenses/}
  \end{itemize}
\end{frame}
%%%%%%%%%%%%%%%%

\input{conteudo/teste}

% ----------------- Referências --------------------------------
\section{Referências}

% --- O comando \allowframebreaks ---
% Se o conteúdo não se encaixa em um quadro, a opção allowframebreaks instrui
% beamer para quebrá-lo automaticamente entre dois ou mais quadros,
% mantendo o frametitle do primeiro quadro (dado como argumento) e acrescentando
% um número romano ou algo parecido na continuação.

\begin{frame}{Referências}
  \bibliography{editaveis/bibliografia}
\end{frame}

% ----------------- FIM DO DOCUMENTO -----------------------------------------

{\aauwavesbg%
\begin{frame}[plain,noframenumbering]%
  \finalpage{Thank you for using this theme!}
\end{frame}}
%%%%%%%%%%%%%%%%

\end{document}
