\documentclass[10pt]{beamer}
\usetheme[
%%% options passed to the outer theme
%    hidetitle,           % hide the (short) title in the sidebar
%    hideauthor,          % hide the (short) author in the sidebar
%    hideinstitute,       % hide the (short) institute in the bottom of the sidebar
%    shownavsym,          % show the navigation symbols
%    width=2cm,           % width of the sidebar (default is 2 cm)
%    hideothersubsections,% hide all subsections but the subsections in the current section
%    hideallsubsections,  % hide all subsections
%    left                % right of left position of sidebar (default is right)
  ]{Aalborg}

% If you want to change the colors of the various elements in the theme, edit and uncomment the following lines
% Change the bar and sidebar colors:
%\setbeamercolor{Aalborg}{fg=red!20,bg=red}
%\setbeamercolor{sidebar}{bg=red!20}
% Change the color of the structural elements:
%\setbeamercolor{structure}{fg=red}
% Change the frame title text color:
%\setbeamercolor{frametitle}{fg=blue}
% Change the normal text color background:
%\setbeamercolor{normal text}{bg=gray!10}
% ... and you can of course change a lot more - see the beamer user manual.

\usepackage[utf8]{inputenc}
\usepackage[brazil]{babel}
\usepackage[T1]{fontenc}
% Or whatever. Note that the encoding and the font should match. If T1
% does not look nice, try deleting the line with the fontenc.
\usepackage{helvet}

% ---
% Extras
% ---
\usepackage{booktabs}

% colored hyperlinks
\newcommand{\chref}[2]{%
  \href{#1}{{\usebeamercolor[bg]{Aalborg}#2}}%
}

\title[The Aalborg Beamer Theme]% optional, use only with long paper titles
{The Aalborg Beamer Theme}

\subtitle{v.\ 1.3.0}  % could also be a conference name

\date{\today}

\author[Jesper Kjær Nielsen] % optional, use only with lots of authors
{
  Jesper Kjær Nielsen\\
  \href{mailto:jkn@es.aau.dk}{{\tt jkn@es.aau.dk}}
}
% - Give the names in the same order as they appear in the paper.
% - Use the \inst{?} command only if the authors have different
%   affiliation. See the beamer manual for an example

\institute[
%  {\includegraphics[scale=0.2]{aau_segl}}\\ %insert a company, department or university logo
  Dept.\ of Electronic Systems\\
  Aalborg University\\
  Denmark
] % optional - is placed in the bottom of the sidebar on every slide
{% is placed on the bottom of the title page
  Department of Electronic Systems\\
  Aalborg University\\
  Denmark

  %there must be an empty line above this line - otherwise some unwanted space is added between the university and the country (I do not know why;( )
}

% specify the logo in the top right/left of the slide
\pgfdeclareimage[height=1cm]{mainlogo}{figuras/aau_logo_new} % placed in the upper left/right corner
\logo{\pgfuseimage{mainlogo}}

% specify a logo on the titlepage (you can specify additional logos an include them in
% institute command below
\pgfdeclareimage[height=1.5cm]{titlepagelogo}{figuras/aau_logo_new} % placed on the title page
%\pgfdeclareimage[height=1.5cm]{titlepagelogo2}{figuras/aau_logo_new} % placed on the title page
\titlegraphic{% is placed on the bottom of the title page
  \pgfuseimage{titlepagelogo}
%  \hspace{1cm}\pgfuseimage{titlepagelogo2}
}


% ---
% Início do documento
% ---
\begin{document}
% the titlepage
{\aauwavesbg
\begin{frame}[plain,noframenumbering] % the plain option removes the sidebar and header from the title page
  \titlepage
\end{frame}}
%%%%%%%%%%%%%%%%
\input{fixos/agenda}

% ------------------------------------------------------------
% Inserir slides aqui
% ------------------------------------------------------------
\section{Introduction}
% motivation for creating this theme
\begin{frame}{Introduction}{}
\begin{block}{Why the Aalborg beamer theme?}
  \begin{itemize}
    \item<1-> In August 2010, I gave a presentation at the European Signal Processing Conference (EUSIPCO) here in Aalborg. For that purpose, I created the AAU sidebar beamer theme.
    \item<2-> Since there was no Aalborg University (AAU) branded beamer theme, I published the theme after the conference on my website so that other researches and students at AAU could use the theme for their presentations.
    \item<3-> I have received a lot of positive feedback. To my surprise, several people not affiliated with AAU have used the theme despite the heavy AAU branding.
    \item<4-> To make the theme more usable to people not affiliated with AAU, I have decided to create this theme called \alert{Aalborg} which is very similar to the AAU sidebar theme, but it does not come with the heavy AAU branding.
  \end{itemize}
\end{block}
\end{frame}
%%%%%%%%%%%%%%%%

\subsection{License}
% the license
\begin{frame}{Introduction}{License}
  \begin{itemize}
    \item<1-> The AAU logo is covered by copyright rules. I have used the logo from \chref{http://aau.designguides.dk}{http://aau.designguides.dk}. As long as you use the theme for making presentations in connection with your work at AAU, you are allowed to use the AAU logo.
    \item<2-> The rest of the theme is provided under the GNU General Public License v. 3 (GPLv3). This basically means that you can redistribute it and/or modify it under the same license. For more information on the GPL license see \chref{http://www.gnu.org/licenses/}{http://www.gnu.org/licenses/}
  \end{itemize}
\end{frame}
%%%%%%%%%%%%%%%%

\section{Site Móvel}

\begin{frame}{Site Móvel}
\begin{block}{Site móvel \emph{versus} Site completo}
  \begin{itemize}

    \item<1->[] Em 2009...
      \begin{table}
        \centering
        \begin{tabular}{ll}
          \toprule
          \textbf{Abordagem}                & \textbf{Taxa de sucesso}\\
          \midrule
          Site exclusivo para celulares     & 64\% \\
          Site completo para celulares      & 53\% \\
          \bottomrule
        \end{tabular}
        \parbox{0.70\textwidth}{\caption{Estudo conduzido em 2009.}}
      \end{table}

    \item<2->[] Em 2011...
      \begin{table}
        \centering
        \begin{tabular}{ll}
          \toprule
          \textbf{Abordagem}                & \textbf{Taxa de sucesso}\\
          \midrule
          Site exclusivo para celulares     & 64\% \\
          Site completo para celulares      & 60\% \\
          \bottomrule
        \end{tabular}
        \parbox{0.70\textwidth}{\caption{Estudo conduzido em 2011.}}
      \end{table}
  \end{itemize}
\end{block}
\end{frame}
%%%%%%%%%%%%%%%%

\begin{frame}{Site Móvel}
\begin{block}{Principais diretrizes para \emph{sites} móveis}
  \begin{itemize}
    \item<1-> Crie um \emph{site} móvel se tiver recursos para tanto.
    \item<2-> Considere a criação de um \emph{app} para o seu \emph{site}.
    \item<3-> Certifique-se que seus usuários sejam direcionados corretamente para o \emph{site} móvel.
    \item<4-> Ofereça um \emph{link} vísivel para seus usuários conseguirem alternar entre o \emph{site} móvel e \emph{site} completo.
    \item<5-> Considere também a criação de \emph{sites} para dispositivos com telas médias (\emph{tablets} de 6") e grandes (\emph{tablets} de 10").
  \end{itemize}
\end{block}
\end{frame}
%%%%%%%%%%%%%%%%

\begin{frame}{Site Móvel}
\begin{block}{Principais diretrizes para \emph{sites} otimizados para dispositivos móveis}
  \begin{itemize}
    \item<1-> Elimine opções e funcionalidades que não sejam fundamentais.
    \item<2-> Reduza conteúdos, e transfira informações secundárias para páginas secundárias.
    \item<3-> Reduze sua estrutura de navegação ao máximo.
    \item<4-> Amplie os elementos de \emph{interface} facilitando seu clique via \emph{touchscreen}.
  \end{itemize}
\end{block}
\end{frame}
%%%%%%%%%%%%%%%%

\begin{frame}{Site Móvel}
\begin{block}{O primeiro argumento falho}
  \begin{itemize}
    \item<1->[]
      \begin{beamercolorbox}[sep=1em]{alert}
        \textbf{``Eliminar conteúdo ou recursos inevitalvelmente vai desapontar algumas pessoas, portanto é melhor servir o \emph{site} completo sempre.''}
      \end{beamercolorbox}
    \item<2-> Para a maioria das tarefas, os usuários móveis conseguirão uma experiência amplamente melhor de um \emph{site} móvel bem planejado do que de um \emph{site} completo.
    \item<3-> Para uma pequena minoria de tarefas, os usuários serão ligeiramente retardados pelo clique extra para o \emph{site} completo.
  \end{itemize}
\end{block}
\end{frame}
%%%%%%%%%%%%%%%%

\begin{frame}{Site Móvel}
\begin{block}{O segundo argumento falho}
  \begin{itemize}
    \item<1->[]
      \begin{beamercolorbox}[sep=1em]{alert}
        \textbf{``É melhor simplesmente otimizar todo o \emph{site} para o dispositivo móvel desde o início, o que elimina os problemas ao se visualizar o \emph{site} completo em dispositivos móveis.''}
      \end{beamercolorbox}
    \item<2-> Essa abordagem prejudica o usuário \emph{desktop}.
    \item<3-> Os \emph{sites} recebem substancialmente mais tráfego de usuários \emph{desktop}.
    \bigskip
    \item<4->[]
    \begin{beamercolorbox}[sep=1em]{postit}
      A interface com o usuário na plataforma \emph{desktop} difere da interface com o usuário na plataforma móvel de várias maneiras. Ambas plataformas precisam de um projeto próprio.
    \end{beamercolorbox}
  \end{itemize}
\end{block}
\end{frame}
%%%%%%%%%%%%%%%%

\section{Projeto Responsivo}

\begin{frame}{Projeto Responsivo}
\begin{block}{Como funciona?}
  \begin{itemize}
    \item<1-> Os diferentes elementos da página são dispostos em uma grade flexível que se ajusta às dimensões da tela.
    \item<2-> O \emph{layout} multicolunas do \emph{site} para \emph{desktop} se torna um \emph{layout} de uma coluna no dispositivo móvel.
    \bigskip
    \item<3->[] \textbf{O mesmo conteúdo e recursos do \emph{site} são apresentados tanto na versão para \emph{desktop} quanto na versão movel.}
  \end{itemize}
\end{block}
\end{frame}
%%%%%%%%%%%%%%%%

\begin{frame}{Projeto Responsivo}
\begin{block}{Vantagens}
  \begin{itemize}
    \item<1-> Custo de manutenção reduzido.
    \item<2-> Funciona bem para aqueles \emph{sites} nos quais todos os recursos presentes no \emph{site} completo têm igual tendência de serem acessados no dispositivo móvel.
  \end{itemize}
\end{block}
\end{frame}
%%%%%%%%%%%%%%%%

\begin{frame}{Projeto Responsivo}
\begin{block}{Desvantagens}
  \begin{itemize}
    \item<1-> Não envolve a criação de IUs (interfaces de usuário) suficientemente distintas que o uso de cada plataforma requer.
    \item<2-> As diferenças de projeto móvel comparado com \emph{desktop} vão muito além das questões de \emph{layout}.
    \item<3-> A partir do momento que todas as diferenças importantes das plataformas são acomodadas em um \emph{layout} responsivo, voltamos ao ponto inicial: \textbf{dois projetos separados}.
    \item<4-> Usar um projeto \emph{web} responsivo para deixar o site completo acessível nos dispositivos móveis frequentemente resulta em uma experiência de usário móvel abaixo do padrão.
  \end{itemize}
\end{block}
\end{frame}
%%%%%%%%%%%%%%%%

\section{Apps}

\begin{frame}{Apps}
\begin{block}{Sites móveis vs apps}
  \begin{itemize}
    \item<1-> Deve-se fazer tudo especial para o dispositivo móvel desde o início?
      \begin{itemize}
        \item<2-> Algumas empresas nunca receberão um uso móvel substancial.
      \end{itemize}
    \item<3-> Deve-se produzir um \emph{site} móvel ou desenvolver \emph{apps} especiais para os dispositivos móveis?
      \begin{itemize}
        \item<4-> Não há dúvidas, se puder, forneça \emph{apps} móveis!
          \begin{table}
            \centering
            \begin{tabular}{ll}
              \toprule
              \textbf{Abordagem}                & \textbf{Taxa de sucesso}\\
              \midrule
              Site Móvel     & 64\% \\
              Apps           & 74\% \\
              \bottomrule
            \end{tabular}
            \parbox{0.50\textwidth}{\caption{Taxas de sucesso: \emph{apps} vs \emph{sites} móveis. Estudos conduzidos em 2011.}}
          \end{table}
      \end{itemize}
  \end{itemize}
\end{block}
\end{frame}
%%%%%%%%%%%%%%%%

\begin{frame}{Apps}
\begin{block}{Vantagens}
  \begin{itemize}
    \item<1-> Um \emph{app} pode objetivar as limitações e habilidades específicas de cada dispositivo muito melhor do que um \emph{site} dentro de um navegador.
    \item<2-> A superioridade do aplicativo nativo tende a se manter para qualquer plataforma, inclusive computadores \emph{desktop}.
    \item<3-> Rodar código nativo em vez de fazer \emph{download} de páginas da \emph{web} faz diferença numa realidade em que a capacidade de processamento é relativamente melhor do que a velocidade de \emph{internet}.
    \item<4-> Ideal para tarefas que sejam aplicações intensamente ricas em recursos, possuindo intenso manuseio de dados.
  \end{itemize}
\end{block}
\end{frame}
%%%%%%%%%%%%%%%%

\begin{frame}{Apps}
\begin{block}{Desvantagens}
  \begin{itemize}
    \item<1-> O usuário tende a pesquisar sobre determinado assunto primeiramente na \emph{web}.
    \item<2-> A integração entre \emph{sites} móveis é mais simples do que entre \emph{apps}.
    \item<3-> Estar preso às condições das lojas de aplicativos, que retém parte do lucro das transações e censuram determinados conteúdos.
    \item<4-> Os usuários frequentemente pedem o interesse por um \emph{app}.
    \item<5-> Alta dependência da plataforma móvel.
  \end{itemize}
\end{block}
\end{frame}
%%%%%%%%%%%%%%%%

\begin{frame}{Apps}
\begin{block}{Em longo prazo...}
  \begin{itemize}
    \item<1-> As tecnologias \emph{web} melhorarão substancialmente as capacidades do \emph{site} móvel.
    \item<2-> A \emph{internet} tenderá a ser mais rápida do que a evolução da capacidade de processamento efetiva dos celulares.
    \item<3-> A diversificação de plataformas móveis aumentará muito, fragmentando o mercado.
  \end{itemize}
\end{block}
\end{frame}
%%%%%%%%%%%%%%%%

\section{Conclusão}

\begin{frame}{Conclusão}
\begin{block}{Quando a estratégia móvel mudará?}
  \begin{itemize}
    \item<1-> Não existe resposta precisa para essa pergunta...
    \item<2->[] ...mas essa mudança será lenta.
    \bigskip
    \item<3->[]
      \begin{beamercolorbox}[sep=1em]{postit}
        Portanto, se você leva a sério a criação da melhor experiência de usuário móvel, opte pelo desenvolvimento de \emph{apps}.
      \end{beamercolorbox}
  \end{itemize}
\end{block}
\end{frame}
%%%%%%%%%%%%%%%%

\begin{frame}{Conclusão}
\begin{block}{Segundo pesquisas realizadas por Nielsen e Budiu}
  \begin{itemize}
    \item<1-> Em 2000 a \emph{web} móvel era igual a \emph{web desktop} de 1994.
    \item<2-> Em 2009 as expectativas eram de melhoras razoáveis...
    \item<3->[] no entanto...
    \item<4->[]
      \begin{table}
        \resizebox{0.90\textwidth}{!}{%
        \begin{tabular}{p{5cm}lp{2.8cm}}
          \toprule
          \textbf{Tarefa}  & \textbf{Telefones WAP (2000)} &  \textbf{Telefones\break Modernos (2009)} \\
          \midrule
          Encontre a previsão do tempo local para hoje à noite  & 164 seg   & 247 seg   \\
          \midrule
          Encontre o que passará na BBC TV 1 hoje às 8 da noite & 159 seg   & 199 seg   \\
          \bottomrule
        \end{tabular}}
        \parbox{0.90\textwidth}{\caption{Tempos médios das tarefas em 2000 e 2009.}}
      \end{table}
    \item<5->[]
      \begin{beamercolorbox}[sep=1em]{postit}
        38\% mais tempo nessas duas tarefas em 2009 do que em 2000.
      \end{beamercolorbox}


  \end{itemize}
\end{block}
\end{frame}
%%%%%%%%%%%%%%%%

\begin{frame}{Conclusão}
\begin{block}{Os dispositivos móveis modernos são piores que os antigos? \break \break A usabilidade móvel dos sites caiu?}
  \begin{itemize}
    \bigskip
    \item<1-> \textbf{Definitivamente, não!}
    \item<2-> Os telefones e \emph{sites} são definitivamente melhores agora.
    \item<3-> Foi o ambiente de utilização que mudou.
    \item<4-> Os dispositivos antigos eram limitados, porém permitiam acesso rápido ao que ofereciam.
    \item<5-> Agora os usuários são altamente domindaos pelas buscas.
    \item<6-> Consegue-se fazer praticamente tudo, mas de forma mais lenta.
  \end{itemize}
\end{block}
\end{frame}
%%%%%%%%%%%%%%%%

\begin{frame}{Conclusão}
\begin{block}{}
  \begin{itemize}
    \item<1->[]
      \begin{beamercolorbox}[sep=1em]{postit}
        O fato de que fazer a maioria das tarefas leva muito tempo enfatiza ainda mais a necessidade de projetos de \emph{sites} móveis mais simplificados.
      \end{beamercolorbox}
    \bigskip
    \item<2->[]
      \begin{beamercolorbox}[sep=1em]{postit}
        Embora os dispositivos móveis estejam ficando melhores, os grandes avanços precisam vir dos \emph{sites}.
      \end{beamercolorbox}
    \bigskip
    \item<3- >[]
      \begin{beamercolorbox}[sep=1em]{postit}
        Deve-se desenvolver projetos especializados que otimizem a experiência de usuário móvel.
      \end{beamercolorbox}
  \end{itemize}
\end{block}
\end{frame}
%%%%%%%%%%%%%%%%


% ----------------- Referências --------------------------------
\section{Referências}

% --- O comando \allowframebreaks ---
% Se o conteúdo não se encaixa em um quadro, a opção allowframebreaks instrui
% beamer para quebrá-lo automaticamente entre dois ou mais quadros,
% mantendo o frametitle do primeiro quadro (dado como argumento) e acrescentando
% um número romano ou algo parecido na continuação.
\nocite{*}
\begin{frame}{Referências}
  \bibliography{editaveis/bibliografia}
\end{frame}

% ----------------- FIM DO DOCUMENTO -----------------------------------------

{\aauwavesbg%
\begin{frame}[plain,noframenumbering]%
  \finalpage{Obrigado!}
\end{frame}}
%%%%%%%%%%%%%%%%

\end{document}
