\section{Site Móvel}

\begin{frame}{Site Móvel}
\begin{block}{Site móvel \emph{versus} Site completo}
  \begin{itemize}

    \item<1->[] Em 2009...
      \begin{table}
        \centering
        \begin{tabular}{ll}
          \toprule
          \textbf{Abordagem}                & \textbf{Taxa de sucesso}\\
          \midrule
          Site exclusivo para celulares     & 64\% \\
          Site completo para celulares      & 53\% \\
          \bottomrule
        \end{tabular}
        \parbox{0.70\textwidth}{\caption{Estudo conduzido em 2009.}}
      \end{table}

    \item<2->[] Em 2011...
      \begin{table}
        \centering
        \begin{tabular}{ll}
          \toprule
          \textbf{Abordagem}                & \textbf{Taxa de sucesso}\\
          \midrule
          Site exclusivo para celulares     & 64\% \\
          Site completo para celulares      & 60\% \\
          \bottomrule
        \end{tabular}
        \parbox{0.70\textwidth}{\caption{Estudo conduzido em 2011.}}
      \end{table}
  \end{itemize}
\end{block}
\end{frame}
%%%%%%%%%%%%%%%%

\begin{frame}{Site Móvel}
\begin{block}{Principais diretrizes para \emph{sites} móveis}
  \begin{itemize}
    \item<1-> Crie um \emph{site} móvel se tiver recursos para tanto.
    \item<2-> Considere a criação de um \emph{app} para o seu \emph{site}.
    \item<3-> Certifique-se que seus usuários sejam direcionados corretamente para o \emph{site} móvel.
    \item<4-> Ofereça um \emph{link} vísivel para seus usuários conseguirem alternar entre o \emph{site} móvel e \emph{site} completo.
    \item<5-> Considere também a criação de \emph{sites} para dispositivos com telas médias (\emph{tablets} de 6") e grandes (\emph{tablets} de 10").
  \end{itemize}
\end{block}
\end{frame}
%%%%%%%%%%%%%%%%

\begin{frame}{Site Móvel}
\begin{block}{Principais diretrizes para \emph{sites} otimizados para dispositivos móveis}
  \begin{itemize}
    \item<1-> Elimine opções e funcionalidades que não sejam fundamentais.
    \item<2-> Reduza conteúdos, e transfira informações secundárias para páginas secundárias.
    \item<3-> Reduze sua estrutura de navegação ao máximo.
    \item<4-> Amplie os elementos de \emph{interface} facilitando seu clique via \emph{touchscreen}.
  \end{itemize}
\end{block}
\end{frame}
%%%%%%%%%%%%%%%%

\begin{frame}{Site Móvel}
\begin{block}{O primeiro argumento falho}
  \begin{itemize}
    \item<1->[]
      \begin{beamercolorbox}[sep=1em]{alert}
        \textbf{``Eliminar conteúdo ou recursos inevitalvelmente vai desapontar algumas pessoas, portanto é melhor servir o \emph{site} completo sempre.''}
      \end{beamercolorbox}
    \item<2-> Para a maioria das tarefas, os usuários móveis conseguirão uma experiência amplamente melhor de um \emph{site} móvel bem planejado do que de um \emph{site} completo.
    \item<3-> Para uma pequena minoria de tarefas, os usuários serão ligeiramente retardados pelo clique extra para o \emph{site} completo.
  \end{itemize}
\end{block}
\end{frame}
%%%%%%%%%%%%%%%%

\begin{frame}{Site Móvel}
\begin{block}{O segundo argumento falho}
  \begin{itemize}
    \item<1->[]
      \begin{beamercolorbox}[sep=1em]{alert}
        \textbf{``É melhor simplesmente otimizar todo o \emph{site} para o dispositivo móvel desde o início, o que elimina os problemas ao se visualizar o \emph{site} completo em dispositivos móveis.''}
      \end{beamercolorbox}
    \item<2-> Essa abordagem prejudica o usuário \emph{desktop}.
    \item<3-> Os \emph{sites} recebem substancialmente mais tráfego de usuários \emph{desktop}.
    \bigskip
    \item<4->[]
    \begin{beamercolorbox}[sep=1em]{postit}
      A interface com o usuário na plataforma \emph{desktop} difere da interface com o usuário na plataforma móvel de várias maneiras. Ambas plataformas precisam de um projeto próprio.
    \end{beamercolorbox}
  \end{itemize}
\end{block}
\end{frame}
%%%%%%%%%%%%%%%%
