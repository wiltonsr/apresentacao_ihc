\section{Projeto Responsivo}

\begin{frame}{Projeto Responsivo}
\begin{block}{Como funciona?}
  \begin{itemize}
    \item<1-> Os diferentes elementos da página são dispostos em uma grade flexível que se ajusta às dimensões da tela.
    \item<2-> O \emph{layout} multicolunas do \emph{site} para \emph{desktop} se torna um \emph{layout} de uma coluna no dispositivo móvel.
    \bigskip
    \item<3->[] \textbf{O mesmo conteúdo e recursos do \emph{site} são apresentados tanto na versão para \emph{desktop} quanto na versão movel.}
  \end{itemize}
\end{block}
\end{frame}
%%%%%%%%%%%%%%%%

\begin{frame}{Projeto Responsivo}
\begin{block}{Vantagens}
  \begin{itemize}
    \item<1-> Custo de manutenção reduzido.
    \item<2-> Funciona bem para aqueles \emph{sites} nos quais todos os recursos presentes no \emph{site} completo têm igual tendência de serem acessados no dispositivo móvel.
  \end{itemize}
\end{block}
\end{frame}
%%%%%%%%%%%%%%%%

\begin{frame}{Projeto Responsivo}
\begin{block}{Desvantagens}
  \begin{itemize}
    \item<1-> Não envolve a criação de IUs (interfaces de usuário) suficientemente distintas que o uso de cada plataforma requer.
    \item<2-> As diferenças de projeto móvel comparado com \emph{desktop} vão muito além das questões de \emph{layout}.
    \item<3-> A partir do momento que todas as diferenças importantes das plataformas são acomodadas em um \emph{layout} responsivo, voltamos ao ponto inicial: \textbf{dois projetos separados}.
    \item<4-> Usar um projeto \emph{web} responsivo para deixar o site completo acessível nos dispositivos móveis frequentemente resulta em uma experiência de usário móvel abaixo do padrão.
  \end{itemize}
\end{block}
\end{frame}
%%%%%%%%%%%%%%%%
