\section{Usabilidade nos Celulares}

\begin{frame}{Usabilidade nos Celulares}{}
\begin{block}{Como era a usabilidade nos celulares de 2009}
  \begin{itemize}
    \item<1-> Usuários não conseguiam executar as tarefas que desejavam.
    \item<2-> Páginas lentas que desmotivavam os usuários.
    \item<3-> Conteúdo dos \emph{sites} não cabiam na tela, deixando os usuários perdidos.
    \item<4-> Alta poluição visual.
    \item<5-> Falta de familiaridade com a interface de usuário.
    \item<6-> Problemas ao visualizar mídias digitais (principalmente \alert{JavaScript}).
    \item<7-> Sites pensados para \emph{desktops}.
    \item<8-> Relutância em realizar diversas tarefas no celular, em especial \alert{compras} e \alert{transações financeiras}.
  \end{itemize}
\end{block}
\end{frame}
%%%%%%%%%%%%%%%%

\begin{frame}{Usabilidade nos Celulares}
\begin{block}{O que melhorou}
  \begin{itemize}
    \item<1-> As taxas de sucesso aumentaram, especialmente nos \emph{apps}.
    \item<2-> Os usuários estão mais familiarizados com seus telefones.
    \item<3-> Aumento da preocupação do funcionamento das mídias digitais e queda do uso de JavaScript.
  \end{itemize}
\end{block}
\end{frame}
%%%%%%%%%%%%%%%%

\begin{frame}{Usabilidade nos Celulares}
\begin{block}{O que ainda é um problema}
  \begin{itemize}
    \item<1-> Relutância em realizar compras via celular melhorou mas ainda persiste.
    \item<2-> Dificuldade em realizar \emph{download} de conteúdos.
    \item<3-> Sites completos continuam difíceis de navegar.
  \end{itemize}
\end{block}
\end{frame}
%%%%%%%%%%%%%%%%

\begin{frame}{Usabilidade nos Celulares}
\begin{block}{Categorias de dispositivos móveis}
  \begin{itemize}
    \item<1-> \textbf{Telefones celulares normais} com tela minúscula, permite apenas uma interação mínima com os \emph{sites}.
    \item<2-> \emph{\textbf{Smartphones}} com tela de tamanho médio e teclado completo(A-Z), possui conectividade 3G e até mesmo Wi-Fi.
    \item<3-> \textbf{Telefones com tela completa} com tela grande e sensível ao toque e GUI(interface gráfica com o usuário) funcional.

    \item<4->[]
      \begin{table}
        \centering
        \begin{tabular}{ll}
          \toprule
          \textbf{Telefone}       & \textbf{Taxa de sucesso}\\
          \midrule
          Telefones com recursos  & 44\% \\
          Smartphones             & 55\% \\
          Telefones de toque      & 74\% \\
          \bottomrule
        \end{tabular}
        \parbox{0.70\textwidth}{\caption{Taxas de sucesso em tarefas para diferentes tipos de telefone ao longo de estudos de teste com usuários, 2009-2012}}
      \end{table}
  \end{itemize}
\end{block}
\end{frame}
%%%%%%%%%%%%%%%%

\begin{frame}{Usabilidade nos Celulares}
\begin{block}{Vale a pena melhorar a usabilidade em telefones com recursos?}
  \begin{itemize}
    \item<1-> A usabilidade do telefone com recursos é muito pobre.
    \item<2-> Empiricamente, observa-se muito pouco tráfego vindo dos telefones com recursos ou mesmo \emph{smartphones}.
    \item<3-> Entre os 3 tipos de telefones citados, o que mais cresce é o de telefones de toque.
    \bigskip
    \item<4->[]
    \begin{beamercolorbox}[sep=1em]{postit}
      Não se justifica o esforço gasto para se criar experiências diferentes para cada tipo de celular. A opção mais realística é criar um site móvel único pensado para os telefones mais modernos.
    \end{beamercolorbox}
  \end{itemize}
\end{block}
\end{frame}
%%%%%%%%%%%%%%%%
